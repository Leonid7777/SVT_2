\documentclass{beamer}
% Преамбула
\title{My Super Topic}
\author{Ashot and Alice}
\usepackage[utf8]{inputenc}
\usepackage[T2A]{fontenc}     
\usepackage[english,russian]{babel} 
\usepackage{amsfonts}
\usepackage{amsmath}
\usepackage{graphicx}
\graphicspath{ {./images/} }

\begin{document}

\begin{frame}
\thispagestyle{empty}
\centerline{Московский государственный университет}
\vfill
\centerline{Факультет вычислительной математики и кибернетики}
\vfill
\vfill
\vfill
\vfill
\Large
\begin{centering}
{\bfseries Численное решение 2D уравнения Лапласа"}
\end{centering}
\normalsize
\vfill
\vfill
\vfill
\begin{flushright}
Студент: Бурцев Леонид\\
Группа: 303
\end{flushright}
\vfill
\vfill
\centerline{Москва}
\centerline{2024}
\pagebreak
\end{frame}

\begin{frame}
\textbf{Описание исходной задачи}
\newline
Необходимо решить краевую задачу Дирихле для уравнения Лапласа

\begin{equation*}
 \begin{cases}
   - \Delta u = f, x \in \Omega = (0, 1)^{2}\\
   u(x) = g, x \in \delta \Omega
 \end{cases}
\end{equation*}
численно с помощью метода конечных разностей.
\newline
Решаем задачу в квадрате $\mathbf{(0, 1)^{2}}$, вводя на ней равномерную сетку ${(x_i, y_i)}$, $i,j = 0, \ldots , N$, где 
$x_i = i * h$,
$y_j = j * h$,
$h = \frac{1}{N}$ - шаг сетки.
\newline
Дискретная аппроксимация уравнения:
$$-\frac{u_{i+1,j}-2u_{i,j}+u_{i-1,j}}{h^{2}} -\frac{u_{i,j+1}-2u_{i,j}+u_{i,j-1}}{h^{2}} = f(x_{i,j})$$
Пятиточечный шаблон приводит к линейной системе с матрицей, в которой 5 ненулевых элементов в строке.
\newline
В граничных узлах значения известны, поэтому неизвестные там не вводятся.
\end{frame}

\begin{frame}
\textbf{Описание численного решения}
\newline
Для решения полученной линейной системы использовался стабилизированный метод бисопряженных градиентов $\textbf{BiCGStab}$ с переобуславливателем $\textbf{MPTILUC}$ из пакета INMOST.

\end{frame}

\begin{frame}
\textbf{Для функции $u=sin(x)sin(5y)$ графики зависимости точности с-нормы и дискретной l2-нормы от шага сетки в логарифмическом виде выглядят следующим образом}

\begin{figure}
    \centering
    \includegraphics[width=0.8\linewidth]{Снимок экрана 2024-03-12 в 21.17.41.png}
\end{figure}
\textbf{Сходимость l2-нормы: $O(0.77 * h^{2.07})$}    
\newline
\textbf{Сходимость c-нормы: $O(h^{2})$} 
\end{frame}

\begin{frame}
\textbf{Графики времени построения решения и времени
итераций от размера системы}  
\begin{figure}
    \centering
    \includegraphics[width=1\linewidth]{Снимок экрана 2024-03-12 в 21.27.17.png}
\end{figure}
\end{frame}
\end{document}

